\documentclass[
    a4paper, %
    stu,     % jou (journal), man (journal submission), stu (student paper), doc (normal)
]{apa7}
\usepackage[british]{babel}
\usepackage{times}
\usepackage{csquotes}

\usepackage[style=apa,backend=biber]{biblatex}
\addbibresource{references.bib}

% ------------------------------------------------------------------------------
%                                  FRONT PAGE
% ------------------
% Title and Authors
\title{Primary Title}
\shorttitle{Short Title}

\leftheader{S{\o}lvsten}

\author{Steffan Christ S{\o}lvsten}
\affiliation{Department of Computer Science, Aarhus University}

% Multiple authors?
% \authorsnames{First1 Surname1, First2 Surname2}
% \authorsaffiliation{...}

% ------------------
% Course information
\course{PSYC 3170: Course Name}
\professor{Teacher's name}
\duedate{\today}

\begin{document}
% ------------------------------------------------------------------------------
% Title Page
\maketitle
% ------------------------------------------------------------------------------
% Content
\section{Introduction}
To write a paragraph, you just write it in the input file, as you normally
would. To start another paragraph, you just need to use a double line break.

For example, here starts a new paragraph.

\subsection{Citations}
\LaTeX\ makes citations really easy. The information is placed in the
\emph{references.bib} file and then you can do a narrative citation, such as
\textcite{soelvsten2022:TACAS}, and a parenthetical citation,
e.g.~\parencite{soelvsten2022:TACAS}. To include page numbers in a citation, you 
an extend the parenthetical citation with optional information,
e.g.~\parencite[p.~64]{soelvsten2022:TACAS}.

\subsection{Quotes}
To add quotations, you want to use the \texttt{blockquote} command which will by
itself make sure whether to show it inline, e.g.\
\blockquote[Albert Einstein]{Two things are infinite: the universe and human
stupidity; and I'm not sure about the universe.}
If the text becomes longer than some threshold, then it is displayed
as an indented block, such as the following
\blockquote[Albert Einstein]{The important thing is not to stop questioning.
Curiosity has its own reason for existence. One cannot help but be in awe when he
contemplates the mysteries of eternity, of life, of the marvelous structure of
reality. It is enough if one tries merely to comprehend a little of this mystery
each day.}
And here the very same paragraph continues.

% ------------------------------------------------------------------------------
% Bibliography
\printbibliography
\end{document}
